\documentclass[10pt,sigconf,letterpaper,nonacm]{acmart}
%\documentclass[10pt,sigconf,anonymous]{sig-alternate-10pt}
\settopmatter{printacmref=false}
\renewcommand\footnotetextcopyrightpermission[1]{} % remove footnote with conference information
\def\baselinestretch{0.98}
\setcopyright{none}
\pagestyle{plain} % remove running headers


\usepackage{subfigure,multirow}
\usepackage{graphicx}
\usepackage{url}
\usepackage{color}
\newcommand{\red}[1]{\textcolor{red}{#1}}
\newcommand{\blue}[1]{\textcolor{blue}{#1}}
\usepackage{array}
\usepackage{amsmath}
\usepackage{autobreak}
\usepackage{indentfirst} 

\newcommand{\tabincell}[2]{\begin{tabular}{@{}#1@{}}#2\end{tabular}}




\begin{document}
\bibliographystyle{plain}
\thispagestyle{empty}

\title{Presentation 1 report}
\author{Yihao Liu, Xiangyu Yin, Kai Huang}
\date{}
\maketitle

\section{Introduction}
Cardiac magnetic resonance imaging (MRI) is a unique non-ionizing radiation technique which provides clear view of heart’s anatomy. The anatomical information of MRI is useful in diagnosing cardiac disease. Whole heart segmentation (WHS) aims to extract the substructures of the heart, commonly including the four chamber blood cavities, the left ventricular myocardium, and sometimes the great vessels as well if they are of interest\cite{zheng2008four}. In order to automatically and accurately extract anatomical information, researchers nowadays are exploring novel algorithms in automatic WHS. 

However, Obtaining fully automatic WHS is arduous due to the low image quality and indistinct boundaries between substructures of the heart in cardiac MRI images. In order to solve this problem, Zhuang et. al\cite{zhuang2016multi} applies multi-modality atlas for whole heart segmentation. They utilize the cardiac CT images to help with the WHS of cardiac MRI images. Since CT is based on a different imaging principle than MRI, CT cardiac image is a good supplementary where the boundary in MRI image is blurry.

Taking advantage of the information shared between mod-alities is beneficial for cardiac disease diagnosis. However, it is necessary to overcome inherent anatomical misregistrations and disparities in signal intensity across the modalities to claim this benefit. Chartsias et.al\cite{chartsias2019multi} proposed a deep learning based method for learning disentangled representations of anatomical and imaging factors from each modality. Shared anatomical factors from different modalities are jointly processed to achieve better segmentation accuracy.

\section{Implementation of multi-modality}
 Figure \ref{yihao_overview} shows the overview of multi-modality model in Zhuang's work\cite{zhuang2016multi}. The segmentation process is based on weighted label fusion(LWF). The key idea is to find the similarity between the training set and test images, and assign labels from maximum likelihood. The training images include MRI and CT images and are manually labeled with different cardiac regions. Since these images are from different individuals or taken at different perspective, each image should be aligned to a standard cardiac model by affine deformation. The label fusion result from each training image is weighted by the similarity to the target image. In order to have a clearer boundary, the test images are converted into entropy atlases using different patch sizes. These atlases are used as the input of a Gaussian regression model to output the mask of the boundary of heart. Finally, by combining the label fusion result and the mask, the target image is segmented by the LWF model.
 
\begin{figure}
	\centering
	\includegraphics[width=0.4\textwidth]{images/yihao_overview.png}
	\caption{Overview of multi-modality model}
	\label{yihao_overview}
\end{figure} 
 
 The entropy atlas are computed as follows\cite{wachinger2012entropy}. For each pixel in a cardiac image, a local patch centered on the pixel is selected. For each of these patches, one can compute its intensity entropy and store this entropy value in the location of the pixel to create a new image. Large patch size suppresses the fin structure but captures global structural information within a certain vision window, while small patch size provides more local details of the structure. The aim of utilizing different patch scales is to get the advantages of both scales.

 Figure \ref{kaihuang_overview} shows the overview of multi-modality model in Chartsias's work \cite{chartsias2019multi}. Their model achieves higher segmentation accuracy by mapping multi-modal images of the same object into disentangled anatomical and modality factors, through an image reconstruction process. Anatomical factors are represented as categorical feature maps. Each category corresponds to input pixels that are, ideally, spatially similar, and hence belong to the same anatomical part. This insight helps promote semantic consistency, and also the learning of spatial correspondences between modalities. Modality factors encode pixel intensity information in a smooth multivariate Gaussian manifold by leveraging Variational Autoencoder (VAE) \cite{KingmaW13}. Anatomical factors are used to obtain segmentation masks, whereas their re-entanglement with the modality factors can be used for image reconstruction.
 
\begin{figure}
	\centering
	\includegraphics[width=0.4\textwidth]{images/kaihuang_overview.png}
	\caption{Overview of multi-modality model}
	\label{kaihuang_overview}
\end{figure} 
 When learning with multiple modalities, anatomical factors obtained from multi-modal images are co-registered with a Spatial Transformer Network (STN) \cite{JaderbergSZK15}, fused with feature arithmetic, and also decoded in different modalities as defined by the modality factors.
 
 
 \section{Discussion}
 The model in Zhuang et.al's work\cite{zhuang2016multi} is simple compared with complex data driven deep neural network. From our understanding, the key factor that affects the performance of WHS is whether we can obtain a clear boundary in MRI image. Thus, a GAN model is a more reasonable direction in multi-modality, whose aim is to convert the low quality cardiac images into high quality level.
 
 The model in Chartsias et.al's work \cite{chartsias2019multi} is dealing with spatial domain representation of multi-modal data. The modality factor that is useless to segmentation can be interpreted by low frequency components in frequency domain. In order to strengthen the anatomical factor and weaken the signal intensity information, designing a filtering process in frequency domain is worthwhile to further improve the extent of disentanglement.



\setcounter{page}{1}

\bibliography{bibfile}
\end{document}


